%% 
%% This is a sample doctoral dissertation.  It shows the appropriate
%% structure for your dissertation.  It should handle most of the
%% strange requirements imposed by the Grad school; like the different
%% handling of titles of one/many appendices.  It will automatically
%% handle the linespacing changes.  The body default is double-spaced
%% (except when you use the singlespace or condensed options).  The
%% default for quotations is single-space, and the default for tabular
%% environments is also single-space.  
%%
%% This class adds the following commands and environments to the
%% report class, upon which it is based:
%% Commands
%% ------------
%% \degree{name}{abbrv} -- Sets the name and abbreviation for the degree.
%%                         These default to ``Doctor of Philosopy''
%%                         and ``Ph.D.'', respectively.
%% \copyrightyear{year} -- for the copyright page.
%% \bachelors{degree}{institution} -- for the abstract
%% \masters{degree}{institution}   --  "
%%     if you have other degrees you may use
%% \secondbachelors{degree}{institution}
%% \thirdbachelors{degree}{institution}
%% \secondmasters{degree}{institution}
%% \thirdmasters{degree}{institution}
%% \priordoctorate{degree}{institution}
%%
%% \committeechair{name}           -- for the signature page
%% or, if you have two co-chairs:
%% \cochairs{first name}{second name}
%%
%% \firstreader{name}              --  "
%% \secondreader{name}             --  "
%% \thirdreader{name}              -- (optional)
%% \fourthreader{name}             --  "
%% \fifthreader{name}              --  "
%% \sixthreader{name}              --  "
%% \departmentchair{name}          -- for the signature page
%% \departmentname{name}           --  "
%%
%% \copyrightpage                  -- produces the copyright page
%% \signaturepage                  -- produces the signature page
%%
%% \frontmatter                    -- these are required in their various
%% \mainmatter                     -- appropriate locations
%% \backmatter                     --
%%
%% \unnumberedchapter[toc]{name}   -- like \chapter, except that it
%%                                    produces an unnumbered chapter;
%%                                    alternatively, like \chapter*,
%%                                    except that it lists the chapter
%%                                    in the table of contents.
%%
%% New environments:
%%   dedication  -- for the dedication
%%   abstract    -- for the abstract
%%
%% The thesis documentclass is built on top of the report document class.
%% It accepts all of the options that the report class accepts, plus the
%% following:
%%     doublespace -- the default, indicates double spacing as per U.Mass.
%%                    requirements.  You will need this when you do your
%%                    final copy.
%%     singlespace -- for earlier work, not acceptable to the Grad school
%%     condensed   -- for earlier work, not acceptable to the Grad school,
%%                    creates condensed versions of the frontmatter. 
%%                    Condensed implies singlespace.
%%     dissertation - the default, indicates that this document is a
%%                    dissertation.
%%     proposal    -- indicates that this document is a dissertation proposal,
%%                    rather than a dissertation.  This will only change the
%%                    wording on the title and signature pages.
%%     thesis      -- indicates that this document is a Master's thesis 
%%                    rather than a doctoral dissertation.  This also changes
%%                    the default for \degree to Master of Science, M.S.
%%     allowlisthypenation -- (the default), allows hyphenation of words in
%%                    the table of contents, the list of figures, and the list
%%                    of tables.  I believe that this is acceptable to the 
%%                    Graduate School.
%%     nolisthyphenation -- disallows hyphenation of words in the table of
%%                    contents and the list of figures and tables.  Use this 
%%                    option if the Grad School doesn't like your hyphenation.
%%     nicerdraft  -- relaxes some of the Grad School's rules for working with
%%                    drafts -- has no effect when doublespace is in effect
%%     nonicerdraft -- the default, leaves things in draft as they will be in
%%                     the final version
%% umassthesis changes the default font size to 12pt, but you may specify 10pt or
%%   11pt in the options.
\documentclass{umassthesis}          % for Ph.D. dissertation or proposal
% \documentclass[thesis]{umassthesis}  % for Master's thesis

%%
%% If you have enough figures or tables that you run out of space for their
%% numbers in the List of Tables or List of figures, you can use the following
%% command to adjust the space left for numbers.  The default is shown:
%%
%% \setlength{\tablenumberwidth}{2.3em}

%% Use the hyperref package if you're producing a version for online
%% distribution and you want hyperlinks.  Note that the Grad School doesn't want
%% their PDF viewers to colorize or otherwise highlight the links; use the
%% hidelinks option to hyperref to avoid decorating links.
%\usepackage[hidelinks]{hyperref}

%% One way of formatting the epigraph/frontispiece is to use this package.
%\usepackage{epigraph}
\usepackage{amsmath}

\begin{document}

%%
%% You must fill in all of these appropriately
\title{Infectious Disease Forecasting for Public Health}
\author{Graham Gibson}
\date{August 2020} % The date you'll actually graduate -- must be
                     % February, May, or September
\copyrightyear{2020}
\bachelors{B.A.}{University of Chicago}

% \committeechair{B. B. Bahh}
\committeechair{Nicholas G. Reich}
\firstreader{Laura Balzer}
\secondreader{Dave Osthus}

%\fifthreader{}            % Optional
%\sixthreader{}            % Optional
\departmentchair[Chair of the Faculty]{Pete Shearer} % Default uses "Department Chair" as the title. To
% use an alternate title, such as "Chair", use \departmentchair[Chair]{Pete Shearer}
% CICS uses "Chair of the Faculty" as of 2019.
\departmentname{Biostatistics \& Epidemiology}

%% If your degree is something other than a Ph.D. (for a dissertation), or
%% an M.S. (for a thesis), you will need to uncomment the appropriate
%% following line:
%%
%% \degree{Doctor of Education}{Ed.D.}
\degree{Doctor of Philosophy}{Ph.D.}
%%
%% \degree{Master of Arts}{M.A.}
%% \degree{Master of Arts in Teaching}{M.A.T.}
%% \degree{Master of Business Administration}{M.B.A.}
%% \degree{Master of Education}{M.Ed.}
%% \degree{Master of Fine Arts}{M.F.A.}
%% \degree{Master of Landscape Architecture}{M.L.A.}
%% \degree{Master of Music}{M.M.}
%% \degree{Master of Public Administration}{M.P.A.}
%%\degree{Master of Public Health}{M.P.H.}
%% \degree{Master of Regional Planning}{M.R.P.}
%% \degree{Master of Science}{M.S.}
%% \degree{Master of Science in Accounting}{M.S. Acctg.}
%% \degree{Master of Science in Chemical Engineering}{M.S. Ch.E.}
%% \degree{Master of Science in Civil Engineering}{M.S.C.E.}
%% \degree{Master of Science in Electrical and Computer Engineering}{M.S.E.C.E.}
%% \degree{Master of Science in Engineering Management}{M.S. Eng. Mgt.}
%% \degree{Master of Science in Environmental Engineering}{M.S. Env. E.}
%% \degree{Master of Science in Industrial Engineering and Operations Research}{M.S.I.E.O.R.}
%% \degree{Master of Science in Manufacturing Engineering}{M.S. Mfg. Eng.}
%% \degree{Master of Science in Mechanical Engineering}{M.S.M.E.}
%%
%% \degree{Professional Master of Business Administration}{P.M.B.A.}


%%
%% These lines produce the title, copyright, and signature pages.
%% They are Mandatory; except that you could leave out the copyright page
%% if you were preparing an M.S. thesis instead of a PhD dissertation.
\frontmatter
\maketitle
\copyrightpage     %% not required for an M.S. thesis
\signaturepage

%%
%% Dedication is optional -- but this is how you create it
%%\begin{dedication}              % Dedication page
  %%\begin{center}
    %%\emph{To those little lost sheep.}
  %%\end{center}
%%\end{dedication}

%%
%% Epigraph (aka frontispiece) is also optional, but this is one way you
%% can create it
%\begin{frontispiece}
%  %% Format to your liking -- see documentation of epigraph package
%  \setlength{\epigraphrule}{0pt}
%
%  \begin{epigraphs}
%    \qitem{%
%      \itshape
%      Mary had a little lamb,\\
%      Her fleece was white as snow.\\
%      \vspace{\baselineskip}
%      And everywhere that Mary went\\
%      The lamb was sure to go.
%      \vspace{\baselineskip}}
%    {Sarah Josepha Hale}
%
%    \vspace{2\baselineskip}
%    \qitem{%
%      \itshape
%      Baa, baa, black sheep,\\
%      Have you any wool?\\
%      Yes, sir, yes, sir,\\
%      Three bags full;\\
%      One for the master,\\
%      And one for the dame,\\
%      And one for the little boy\\
%      Who lives down the lane.
%      \vspace{\baselineskip}}
%    {English Nursery Rhyme}
%
%  \end{epigraphs}
%\end{frontispiece}

%%
%% Acknowledgements are optional...yeah, right.
\chapter{Acknowledgments}             % Acknowledgements page
  I would like to thank my advisor and professors for their guidance and 

%%
%% Abstract is MANDATORY. -- Except for MS theses
\begin{abstract}                % Abstract
  Sheep like grass.  Why?  Let me tell you.  Sheep are ruminants, like
  cattle, deer, and horses.  They have stomachs that are specialized...
\end{abstract}

%%
%% Preface goes here...would be just like Acknowledgements -- optional
%% \chapter{Preface} 
%% ...


%%
%% Table of contents is mandatory, lists of tables and figures are 
%% mandatory if you have any tables or figures; must be in this order.
\tableofcontents                % Table of contents
\listoftables                   % List of Tables
\listoffigures                  % List of Figures

%%
%% We don't handle List of Abbreviations
%% We don't handle Glossary

%%%%%%%%%%%%%%%%%%%%%%%%%%%%%%%%%%%%%%%%%%%%%%%%%%%%%%%%%%%%%%%%%%%%%%%%%
%% Time for the body of the dissertation
\mainmatter   %% <-- This line is mandatory

%%
%% If you want an introduction, which is not a numbered chapter, insert
%% the following two lines.  This is OPTIONAL:
\unnumberedchapter{Introduction}

   Infectious disease modeling has emerged as a powerful public health tool in the last century. In an effort to understand the trajectory, impact, and intervention targets of diseases, modelers from around the globe have turned their attention to infectious diseases. While still in its infancy (compared to weather modeling), infectious disease models have been used to make actionable public health decisions. Most recently, government agencies have begun looking for to infectious disease modeling groups to produce forecasts in real-time. These have included an influenza forecasting challenge hosted by the CDC, a Dengue challenge hosted by NOAA, and most recently a COVID-19 forecasting competition also organized by the CDC. The desire for accurate infectious disease forecasts at the decision making level is clearly increasing. 
   
   This thesis focuses on the use of infectious disease models in real-world forecasting scenarios where data collection is imperfect, model parameters are difficult to identify, and operational forecasts are made with actionable consequences under time constraints. In real epidemic scenarios, such as the COVID-19 pandemic, forecasts must be made at multiple spatial scales. While in theory, modelers should be able to account for spatial correlation in the model, this is not always possible in real-time forecasting scenarios. In chapter one, we explore projection techniques that allow modelers to produce independent forecasts by region while still retaining the underlying spatial hierarchy  of the surveillance process. Another frequent challenge in real-time forecasting is the variety of data quality issues that arise. Data from surveillance systems, especially in an emerging disease such as COVID-19, is often under-reported, mis-reported and delayed. Additionally, when observations are made on multiple data streams, such as cases and deaths, these factors can have different effects on each data stream. Building real-world forecasting models requires taking these operational challenges into consideration when designing the model. In chapter two we describe a novel Bayesian infectious disease forecasting model that accounts for data quality issues, joint observations on cases and deaths, and time-varying transmissibility due to intervention efforts.  The last challenge we address, is the ability to accurately measure the efficacy of interventions put in place during an epidemic. This is particularly important during an emerging epidemic, where the majority of the interventions will be non-pharmaceutical based  (non-vaccination) and will relate more to human mobility (social isolation and quarantining). Understanding the effects of these measures on the time-varying reproduction number (a measure of how many cases a single infected individual is expected to produce at a given time) requires estimation from surveillance data. In chapter three, we explore the properties of common estimators from a probabilistic perspective, to gain insight into their bias in real-world settings. The remainder of the introduction is devoted to providing context for these developments in the larger infectious disease modeling realm.
      
\section{Background}
In 1906, W.H. Hamer proposed adopting an idea familiar in Physics, the law of mass action, to the modeling of epidemics. This idea, translated into modern terminology, reflects the fact that the spread of infection should be related to the number of infected people and the number of susceptible people in a closed population. Models based on the principle were put to use as early as 1911, where Dr. Ross built a compartmental model to describe the dynamics of Malaria. He showed that Malaria could be eradicated from a population, even with the vector present, as long as the number of mosquitos fell below a critical threshold. This lead to actionable public health measures, wehreby mosquito control targets were set based on compartmental model theory. 

The law of mass action led to the development of "compartmental model" theory by Kermack and McCarthy in 1927. However, the model originally proposed is quite different from the standard susceptible-infected-recovered model commonly described in the modern literature. The model they original postulated was defined as 


\begin{eqnarray}
  v(t) &=& -x'(t) \\ 
  x'(t) &=& -x(t)\left[ \int_0^t A(s) v(t-s)ds + A(t) y_0 \right] \\
  z'(t) &=& \int_0^t C(s)v(t-s)ds + C(t)y_0\\
  y(t) &=& \int_0^t B(s)v(t-s)ds + B(t) y_0\\
  B(s) &=& e^{-\int_0^t \psi(s)ds}\\
  A(s) &=& \phi(s)B(s)
\end{eqnarray}

where $x(t)$ is the number of susceptibles at time t, $y(t)$ is the number of infected individuals at time t, and $z(t)$ is the number of recovered individuals, $\phi(s)$ is the recovery rate at age of infection $s$ and $\psi(s)$ is infection rate at age of infection $s$. This model is sometimes referred to as the "age of infection model", since the model requires that we keep track of the time elapsed for each individual, i.e. their "age of infection". This model is heavily based on the differential equation identity commonly known as the Euler Lotka equation from population ecology \cite{wallinga2007generation}. 

\begin{equation}
  b(t) =\int_0^\infty b(t-a)l(a)m(a) da
\end{equation}

where $b(t)$ represents the number of births in a population, $r$ represents the growth rate, $l(a)$ is a function denoting what percentage of offspring survive $a$ time units, $m(a)$ represents the number of offspring a mother will produce at $a$ time units. Kermack and McCormick recognized that this equation could be mapped onto an epidemic by treating new infections as ``births".  We can transform the Kermack and McCormick model to the commonly known SIR differential equation model by applying the following transformation.

\begin{equation}
S(t) = x(t) , \ \ A(s)= B(s) = e^{-\gamma s}, \ \ y(t) =  aI(t)
\end{equation}


\begin{eqnarray}
  S'(t) &=& S(t)\left[ \int_0^t e^{-\gamma s}S'(t-s)ds + e^{-\gamma t} y_0 \right] \\
  R'(t) &=& \int_0^t C(s)v(t-s)ds + C(t)y_0\\
 \beta I (t) &=& -S(t)\int_0^t e^{-\gamma s}S'(t-s)ds + e^{-\gamma t}y_0\\
\end{eqnarray}


We can see, using the renewal equation, that $\S(t)int_0^t e^{-\gamma s}S'(t-s)ds = I(t)$, that these equations reduce to the kernel of the common sir model,


\begin{eqnarray}
S'(t) &=& -\beta*S(t)I(t)\\
I'(t) &=& \beta*S(t)I(t) - \gamma*I(t)
\end{eqnarray}

Under the constraint that $I(t) + R(t) +S(t) = N$ we can add the corresponding differential equation for the number of recovered, $R'(t) = -\gamma I(t)$. Thus, the SIR model was ``born", and became a leading tool for infectious disease epidemiologists to study properties of epidemics.

One of the most studied quantities to come from the simple SIR model is known as the ``reproduction number" or $R_0$. This quantity can be thought of as the expected number of secondary infections a single infected individual will produce at the start of the epidemic. The qualifier, at the start of the epidemic, is added to remove the influence of a limited susceptible population towards the end. This quantity allows for comparison across diseases. We can see the importance of the $R_0$ threshold value of 1 by re-writing when $S(t) \approx 1$, 

\begin{equation}
I'(t) =(\beta - \gamma) *I(t)
\end{equation}

Therefore, if $\beta-\gamma \leq 0$ (which is the same condition as $\frac{\beta}{\gamma}  < 1$the derivative of prevalent infections is negative, meaning infections are decreasing. If $\frac{\beta}{\gamma}  >1$ then infections are increasing. 

We can extend this concept to be time-varying, be defining $R(t)$ to be the expected number of secondary cases an infected individual will produce at time $t$. In the simple SIR model, this allows us to keep track of the diminishing susceptible population. 

\begin{equation}
R_{SIR}(t) = \frac{\beta}{\gamma}S(t)
\end{equation}

Thus, we can see that as the epidemic progresses $R(t)$ is uniformly decreasing due to a uniformly decreasing susceptible population. In particular, if we knew the analytical form for $S(t)$ we could solve for the point at which $R_{SIR}(t) <1$ meaning that the epidemic is ``decreasing".

The final properties of compartmental models we introduce is that of the generation interval and serial interval. The generation interval can be defined as the time from exposure of an individual to secondary infection (when the individual infects someone else). Similarly, the serial interval can be defined as the time from onset of symptoms to secondary infection. Note that in an SIR model, symptom onset and infection occur simultaneously, so the serial interval and generation interval are equal by definition. However, this is not true of all compartmental models. In particular, the susceptible-exposed-infected-recovered (SEIR) model, which adds an E compartment for individuals who have been exposed but have not developed symptoms, has a larger generation interval than serial interval. Returning to the renewal equation setup, we can see that the generation interval can play the role of the time to offspring. Indeed Fraser defines the infectious disease version of the Euler-Lotka equation as \cite{fraser2007estimating}

\begin{equation}
I'(t) = R_t \int_0^t I'(\tau) g_{\tau} d\tau
\end{equation}

where $I'(t)$ denotes the number of new infections at time $t$ and $g_{\tau}$ represents the serial interval probability function evaluated at $\tau$. Here we can think of $\tau$ as a time index from start of epidemic until current time.

Finally, we turn to stochastic compartmental models. First developed in the 1950's by Bailey and Whittle \cite{whittle1955outcome}\cite{bailey1953total}. These models re-framed the differential equation models of 1920's as Markov chains, with transition probabilities that involve the same parameters as the differential equations. We can define the Markov chain analogue of the SIR model as follows \cite{allen2010introduction}\cite{allen2000comparison}\cite{brauer2012mathematical}. 


\begin{equation}
    P(S_{t} +k, I_{t} + j | S_{t-1},I_{t-1}) =
    \begin{cases}
    \beta *I_{t-1} * S_{t-1} * \delta t \ \  \ \ \ \ \ \ \ \ \ \ \ \ \  \ \  \ \ \ \ \ \text{for } k=-1,j=1\\ 
      \gamma  *I_{t-1}*\delta t \ \ \ \ \ \ \ \ \ \ \ \ \ \ \ \ \ \ \ \ \ \ \ \ \ \ \  \ \  \ \ \text{for } k=0,j=-1\\
      
       (1 -\beta *I_{t-1} * S_{t-1}  -\gamma *I_{t-1})*\delta t   \ \ \text{for } k=0,j=0\\
    \end{cases}
\end{equation}

Where we choose $\delta t$ small enough that only one of three mutually exclusive events can occur. Either there is a single infection, in which case the number of susceptibles drops by one and the number of infected increases by one, or there is a single recovery, in which case the number of infected drops by one, or no event occurs. Note that we can recover the number of recovered using the population constraint. Choosing $\delta t$ small enough ensures that each transition is a valid probability $<1$. 


\subsection{Statistical models}
\subsubsection{Deterministic}
\subsubsection{Stochastic}


\section{Challenges}
   However, infectious disease modeling still faces many challenges. On the applied side, these range from data quality issues, to parameter identifiability, to computational efficiency. Real-world infectious disease data is often incomplete, with under-reported of cases due to incomplete testing, revisions due to inaccurate diagnoses, and reporting infrastructure delays that cause significant backlog in reporting.  Differential equation models rely on key epidemiological parameters, such as the time to recovery, of a specific disease. Identifying these parameters from time series data only is a difficult task, and often leads to misspecification. In order to avoid this issue, non-parametric extensions to the classical compartmental models allow us soak up deviations in the data from prior estimates of epidemiological parameters. Finally, fitting complex compartmental models requires extensive computational resources, with fitting usually requiring an approximation algorithm such as the Runge-Katta 4th order approximation that is computationally expensive. 
   
   On the theoretical side, these range from how to effectively model interventions, understanding probabilistic disease models, and finding mathematically valid approximations to the common differential equation models. Intervention models have usually included a parametric or non-parametric model of the transmissibility parameter ($\beta$). Parametric models attempt to capture changes in transmissibility due to interventions such as limiting mobility. Non-parametric models are able to capture the aggregate level of interventions without relying on external data. 
   
 \section{Chapters}  
           In this work, we explore these issues by addressing three problems that arise in infectious disease modeling. First, we explore how to remedy the issue of forecasts that are made for independent geographies, as well as an aggregate geography, may be reconciled into a set of "coherent" forecasts, where the sum of the independent geography forecasts is required to be equal to the aggregate geography forecast. In many forecasting competitions, such as the FluSight and COVID-19 competition, forecasts are required for each state, as well as the nation as a whole. If the data-generating process is such that the regional data is aggregated to a national level, then forecasts for regional data should aggregate to the forecast for the nation. However, due to the variability in forecast quality by geography, taking the simple sum of the regional forecasts is not necessarily the optimal method. In chapter 1, we explore methods to aggregate regional forecasts to improve forecast score under probabilistic scoring. In chapter 2, we operationalize classical compartmental models to forecast COVID-19 in the U.S. in real-time. We extend the classic susceptible-exposed-infected-recovered model to include a negative binomial observation model, time varying transmission, and time-varying case detection probability and demonstrate the performance over the baseline model, as well as other models submitted to the COVID-Hub. In chapter 3, we use the Markov chain construction of the SIR model to demonstrate that the serial interval contracts as an epidemic progresses, and use this to estimate and correct for the bias in a common time varying reproduction number estimator. 

%%
%% Some sample text
\chapter{An Introduction to Sheep}
Is there life after sheep?~\cite{xyz}  Yes, I say there is.%\marginpar{Really?}

\section{Pulling the wool over your eyes}

Sheep are fabulous creatures.  The noises they make are truly stupendous
\cite{Bah}.  We also want to refer to figure \ref{fig:circle} here.
Here's some verbatim text to screw us up:

{\small
\begin{verbatim}
xxx := y;
xy := x;
\end{verbatim}
}

\begin{figure}
  \begin{center}
    \begin{picture}(300,200)
      \put(150,100){\circle{150}}
      \put(1,1){\framebox(298,198){}}
    \end{picture}
    \caption{A circle in a square.}\label{fig:circle}
  \end{center}
\end{figure}

\subsection{All about sheep noises}
Lots of text here just to fill up some space so we can be sure that we
really are double-spacing and doing all the other things that might be
necessary in formatting a dissertation to U.Mass. guidelines.  We're
also going to have another figure here, figure \ref{fig:disc}, just
for fun, and to make sure that the list of figures is formatted
correctly.  Now it's time for table \ref{table:somenumbers}.  We
really are going to need a third figure, figure \ref{fig:discs}, two
more tables, table \ref{table:morenumbers} and table
\ref{table:evenmorenumbers} and a fourth figure, figure
\ref{fig:circleanddisc}, just to really make sure.

\begin{figure}
  \begin{center}
    \begin{picture}(300,200)
      \put(150,100){\circle*{150}}
      \put(1,1){\framebox(298,198){}}
    \end{picture}
    \caption{A disc in a square.}\label{fig:disc}
  \end{center}
\end{figure}

\begin{table}[htbp]
  \begin{center}
    \caption{Some numbers.}
    \label{table:somenumbers}
    \begin{tabular}{|r|lll|}
      \hline
      & Minimum & Average & Maximum \\
      Type of Animal & Observed & Observed & Observed \\ \hline
      Cats & 12 & 20 & 24 \\
      Dogs & 20 & 20 & 20 \\ \hline
    \end{tabular}
  \end{center}
\end{table}

\begin{figure}
  \begin{center}
    \begin{picture}(400,200)
      \put(100,100){\circle*{150}}
      \put(300,100){\circle*{150}}
      \put(1,1){\framebox(398,198){}}
    \end{picture}
    \caption{Two discs in a rectangle.}\label{fig:discs}
  \end{center}
\end{figure}

\begin{table}[htbp]
  \begin{center}
    \caption{More numbers.}
    \label{table:morenumbers}
    \begin{tabular}{|r|lll|}
      \hline
      Type of Animal & Arms & Legs & Ears \\ \hline
      Person & 2 & 2 & 2 \\
      Dog & 0 & 4 & 2 \\ \hline
    \end{tabular}
  \end{center}
\end{table}

\begin{table}[htbp]
  \begin{center}
    \caption[Even more numbers; together with a caption long enough to ensure that multi-line caption formatting works correctly.]{Even more numbers; together with a caption long enough to ensure that multi-line caption formatting works correctly.  If you want a shorter caption to appear in the Table of Figures you're going to have to put the shorter caption in the \texttt{[]} as shown in this example.}
    \label{table:evenmorenumbers}

    \begin{tabular}{|r|lll|}
      \hline
      x & 1 & 1 & 1 \\ \hline
      y & 2 & 2 & 2 \\
      z & 3 & 3 & 3 \\ \hline
    \end{tabular}
  \end{center}
\end{table}

\begin{figure}
  \begin{center}
    \begin{picture}(400,200)
      \put(100,100){\circle{150}}
      \put(300,100){\circle*{150}}
      \put(1,1){\framebox(398,198){}}
    \end{picture}
    \caption{A circle and a disc in a square.  We want this caption to
      be very long to ensure that the formatting of very long captions
      is handled correctly.  The case of short captions has already
      been dealt with.}\label{fig:circleanddisc}
  \end{center}
\end{figure}



\subsubsection{Baahs}
\subsection{Even more about sheep noises}
\subsection{And yet more about sheep noises}

\section{What about wolves?}
What about wolves?\footnote{To be fair, some wolves are probably nice\ldots}

\section{What about shepherds?}
What about shepherds?  I don't really know, but I want some text here
to fill things in so that I can verify that everything is OK.%
\footnote{Some shepherds are good, some are bad. The reader is referred
  to Mary and The Boy Who Cried Wolf for further insight into this
  much-debated issue. (This needs to be a very long footnote so we can
  test the spacing between lines on a footnote.)}
\subsection{A subsection}
This is a subsection of the subsection about shepherds.
\subsection{Another subsection}
This is another subsection of that section.
\subsubsection{A subsubsection}
This is a subsubsection of that subsection that will in turn havae a
paragraph with a pair of subparagraphs.  I am aware that I shouldn't
have only one subsubsection in the subsection...
\paragraph{A Paragraph} 
This is the text associated with this paragraph.  I really want enough
text to make it look like a paragraph.  Baah, baah, baah.  Baah, baah,
baah.  Baah, baah, baah.  Baah, baah, baah.  Baah, baah, baah.  Baah,
baah, baah.  Baah, baah, baah.  Baah, baah, baah.  Baah, baah, baah. 
\subparagraph{A Subparagraph} 
This is the text associated with this subparagraph.  Baah, baah, baah.
Baah, baah, baah.  Baah, baah, baah.  Baah, baah, baah.  Baah, baah,
baah.  Baah, baah, baah.  Baah, baah, baah.  Baah, baah, baah. 
\subparagraph{Another Subparagraph}
Better not have subparagraphs without text in them.  Baah, baah, baah.
Baah, baah, baah.  Baah, baah, baah.  Baah, baah, baah.  Baah, baah,
baah.  Baah, baah, baah.  Baah, baah, baah. 
\paragraph{Another Paragraph}
Baah, baah, baah.  Baah, baah, baah.  Baah, baah, baah.  Baah, baah,
baah.  Baah, baah, baah.  Baah, baah, baah.  Baah, baah, baah.  Baah,
baah, baah.  Baah, baah, baah.  Baah, baah, baah.  Baah, baah, baah.
Baah, baah, baah.  Baah, baah, baah.  Baah, baah, baah.  Baah, baah,
baah.  Baah, baah, baah.  Baah, baah, baah.  Baah, baah, baah.  Baah,
baah, baah.  Baah, baah, baah.  Baah, baah, baah.

Baah, baah, baah.  Baah, baah, baah.  Baah, baah, baah.  Baah, baah,
baah.  Baah, baah, baah.  Baah, baah, baah.  Baah, baah, baah.  Baah,
baah, baah.  Baah, baah, baah.  Baah, baah, baah.  Baah, baah, baah.
Baah, baah, baah.  Baah, baah, baah.  Baah, baah, baah.  Baah, baah,
baah.  Baah, baah, baah.  Baah, baah, baah.  Baah, baah, baah.  Baah,
baah, baah.  Baah, baah, baah.  Baah, baah, baah.
\subsubsection{Another Subsubsection}
With some text.  Baah, baah, baah.  Baah, baah, baah.  Baah, baah,
baah.  Baah, baah, baah.  Baah, baah, baah.  Baah, baah, baah.  Baah,
baah, baah.  Baah, baah, baah.  Baah, baah, baah.  Baah, baah, baah. 

\chapter{Sheep and Grass}

\section{Introduction}

Grass is a wonderful food...  Baah, baah, baah.  Baah, baah, baah.
Baah, baah, baah.  Baah, baah, baah.  Baah, baah, baah.  Baah, baah,
baah.  Baah, baah, baah.  Baah, baah, baah.  Baah, baah, baah.  Baah,
baah, baah.  Baah, baah, baah.  Baah, baah, baah.  Baah, baah, baah.
Baah, baah, baah.  Baah, baah, baah.  Baah, baah, baah.  Baah, baah,
baah.  Baah, baah, baah.  Baah, baah, baah.  Baah, baah, baah.  Baah,
baah, baah.  Baah, baah, baah.  Baah, baah, baah.  Baah, baah, baah.
Baah, baah, baah.  Baah, baah, baah.  Baah, baah, baah. 

\chapter{A Wonderfully Long Chapter Title That Is This Long In Order
  to Test the Chapter Heading Stuff}
Note that we shouldn't really have a chapter heading with no body, so
here is a body for this chapter.  Baah, baah, baah.  Baah, baah, baah.
Baah, baah, baah.  Baah, baah, baah.  Baah, baah, baah.  Baah, baah,
baah.  Baah, baah, baah.  Baah, baah, baah.  Baah, baah, baah.  Baah,
baah, baah.  Baah, baah, baah.  Baah, baah, baah.  Baah, baah, baah.
Baah, baah, baah.  Baah, baah, baah.  Baah, baah, baah.  Baah, baah,
baah.  Baah, baah, baah.  Baah, baah, baah.  Baah, baah, baah.  Baah,
baah, baah.  Baah, baah, baah. 

\section{The antidisestablishmentarainism supercalifragilisticexpialidocious longlonglonglonglongword}

A \texttt{quotation}:

\begin{quotation}
Lorem ipsum dolor sit amet, consectetur adipiscing elit. Ut nibh orci, molestie
non vehicula ac, ultricies quis purus. Nunc euismod metus vel nulla sodales quis
tempus nisi varius. Sed ornare pulvinar bibendum. Ut egestas mollis nisi vel
cursus.
\end{quotation}

\dots and a \texttt{quote}:

\begin{quote}
Ut dolor libero, blandit tristique accumsan non, viverra a magna. Sed pretium
sollicitudin neque, sit amet ornare lorem convallis ac. Fusce mollis gravida
aliquam. Nullam vulputate turpis vitae orci porttitor auctor. Donec in auctor
erat.
\end{quote}



%% End of body
%%%%%%%%%%%%%%%%%%%%%%%%%%%%%%%%%%%%%%%%%%%%%%%%%%%%%%%%%%%%%%%%%%%%%%%%%%%%%%%

\appendix
\chapter{THE FIRST APPENDIX TITLE}
...
\chapter{THE SECOND APPENDIX TITLE}
...

%%
%% Beginning of back matter
\backmatter  %% <--- mandatory

%%
%% We don't support endnotes

%%
%% A bibliography is required.
\interlinepenalty=10000  % prevent split bibliography entries
\bibliographystyle{umassthesis}
\bibliography{bib}
\end{document}

%%% Local Variables: 
%%% mode: latex
%%% TeX-master: t
%%% End: 
